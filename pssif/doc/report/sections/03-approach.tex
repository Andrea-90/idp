\chapter{Approach}
\label{chap:approach}

This chapter presents and, to some extent, justifies the approach of the Proof of Concept (PoC) software utility developed in the scope of this interdisciplinary project. A comparison of different transformation methods on an abstract level is provided in \secref{sec:approach:transform}. Thereafter, the chosen approach to the realization of model transformations is defined on an abstract level in \secref{sec:approach:pssif}.

\section{Transformation Methods}
\label{sec:approach:transform}

Provided with the task to transform between different Models, there is an number of possible solutions. These can roughly be categorized into direct and indirect transformation methods.

\subsection{Direct Transformation Methods}

Direct transformation methods are such which define rules for the transcription from source to destination domain-specific language directly, i.e. such methods do not produce a (defined) intermediate result, but rather are always language-specific. Furthermore, it can be differentiated between syntax-dependent and syntax-independent technical solutions for this kind of transformations.

\subsubsection{Syntax-Dependent Transformations}

Syntax-dependent transformations would build upon a technology which is well-suited for processing the syntax of the source and destination domain-specific languages (DSLs). For example, considering DSLs which are both use the Extensible Markup Language (XML) as their concrete syntax, an appropriate transformation technology might be EXtensible Stylesheet Language (XSLT).

\subsubsection{Syntax-Independent Transformations}

Syntax-independent direct transformation methods are a category of methods, which, while still transforming directly from a source DSL to a destination DSL, are not coupled to the concrete syntax of any particular DSL. Such transformation approaches can be realized in a high-level programming language, which relies on a number of serialization and de-serialization components for the transmission of own language-specific data-structures to the respective DSL concrete syntax serializations.

\subsection{Indirect Transformation Methods}

Indirect methods of transformation are methods which rely on a stable and well defined intermediate format. A particular transformation between two DSLs is performed by first transforming from the source DSL into the intermediate language and then transforming from the intermediate language to the target DSL. Once again, from a technological perspective, a differentiation between two categories of intermediate languages can be made: fixed and flexible intermediate languages.

\subsubsection{Fixed Intermediate Language Transformations}

Transformation methods with a fixed intermediate language can be realized in a high-level programming language. In this case the abstract syntax of the intermediate language is directly implemented as a data structure in the programming language. Thus, the intermediate language is described directly with the vocabulary of the programming language in use. Also, in this case, the transformations from and to DSLs can be implemented directly in the syntax of the particular programming language and are, therefore, most likely Turing-complete.

\subsubsection{Flexible Intermediate Language Transformation}

Transformation methods with a flexible intermediate language can also be technologically solved with a high-level programming language. As opposed to the last case, here the intermediate transformation language is not fixed, in the sense that it is not hard-coded. Note that the intermediate language is still likely to be fixed for the scope of a single transformation.

\subsection{Discussion}

In this section the different transformation methods presented above are compared with each other. Let us first consider syntax-dependent transformation methods such as XSLT. On the one hand, this kind of transformations can be advantageous, because of their closeness to the languages at hand. A direct transformation can always be defined to enclose the maximal possible transmittable detail from one DSL to another. On the other hand, such approaches impose a limitation to the entirety of languages which can be supported, due to their binding to the concrete syntax of those languages.

Syntax-independent direct transformations resolve this issue by abstracting the transformation description from the concrete syntax of the particular language, but still have a number of significant disadvantages. The most important one of these is the fact that such transformation methods require an explicit implementation for each pair of source and target DSLs. As a result, with the introduction of each new language, a transformation procedure has to be defined for the combination of this language with every language already supported. Illustratively put, the result is a complete graph (each node being a DSL and each edge a language-to-language transformation) and the number of necessary implementations grows exponentially in the number of DSLs to support. In a dynamic field, where new languages may appear at any time, direct transformation approaches would thus incur significant costs on the longer run. Another disadvantage of these approaches is that with the growing number of transformation implementations, the code-base also grows proportionally. As a result, the code maintenance for such a utility also becomes more costly with time. 

To address the issue of exponentially growing complexity, indirect transformation methods can be used. As already noted, the methods in this category use an intermediate format to and from which transformations are made for each DSL. As a result, the addition of a new DSL will require the additional implementation of just one transformation, as opposed to as many transformations as there are languages to support. The direct solution in this case would be to directly implement the intermediate language in the programming language of choice. Such a description of the intermediate language and its transformation would, on the one hand, have the advantage of being expressed in the concepts of the used programming language directly and thus being accessible to any person with knowledge of this programming language. On the other hand, a fixed intermediate language is likely to be more costly once the evolution of the transformation framework is taken into consideration. In particular, any conceptual change in the intended meaning of a transformation would directly impact the transformations to and from all DSL on the level of the programming language used. In essence, it would be necessary to potentially rewrite the code used for all transformations, which, considering a growing number of DSLs and a continuously evolving modelling methodology, would once again incur long-term costs. Also, while the code base in this case is significantly smaller than in the case of direct transformation approaches, it still is growing in proportion with the number of DSLs to support.

Finally, the most abstract approach is the usage of flexible intermediate language transformations. As opposed to the last approach, here the intermediate language is not bound by concepts of the underlying programming language, but rather is only expressed in those concepts. As a result, the intermediate language can be defined on a level of abstraction on which most, if not all, new requirements can be expressed in terms of instantiation instead of code generation. In particular, only requirements which introduce new concepts in the language would require a modification of the code base. Requirements which merely imply a structural change can the implemented by configuration. The advantages of such an approach are numerous, most importantly that it would provide a viable, flexible and powerful tool with only limited costs for the introduction of new DSLs. The major disadvantage of such an approach is that it incurs a significant initial effort, as it requires the infrastructure for the desciption of the intermediate language to be implemented as well.

With this considerations made, the authors consider a flexible intermediate language approach to be the most appropriate one for the PoC software utility, as it best addresses the purposes of the PSS Integration Framework.

\section{The PSS-IF Transformation Method}
\label{sec:approach:pssif}

In the last section the choice of a flexible intermediate language as a transformation method was made. In such a transformation method, there are a few central concepts, around which further definitions revolve.

\subsection{Levels of Meta}

In the rest of this documentation, the terms Metamodel and Model are used continuously. For this reason, here these are defined somewhat more clearly from the perspective of levels of information and structure. The most basic of those levels is reality itself, which is, as such, incomprehensible in its fullness. This is why a model is defined as:

\paragraph{Model} A simplified description of reality, which only contains certain aspects of it, relevant for the creator of the model.\\

A model, while depicting certain aspects of reality, does not necessarily conform to any particular structure. The structure of a model is provided by a metamodel, defined as:

\paragraph{Metamodel} A description which defines how models are structured. A model is said to be an (ontological) instance of a metamodel, if the model structurally conforms to the metamodel.\\

Note that no constraint is given as to the number of meta-levels involved -- one can easily define meta-metamodels etc.

\subsection{Viewpoints and Views}

In the last section we have defined the terms metamodel and model. In this section, the terms Viewpoint and View are added.

\paragraph{View} A view is a subset of a model.

\paragraph{Viewpoint} A viewpoint is a description which defines how views are created.\\

A viewpoint can thus be seen as a restricted or, more generally, transformed metamodel, which makes it possible to perceive model instances of this metamodel from a certain perspective (viewpoint). More precise definitions of this terms can be found in the ISO 42010 standard.

\subsection{PSS-IF in a Nutshell}
\label{sec:approach:pssif:nutshell}

Consider a metamodel which describes all aspects of reality, which are of importance for the development of a Product-Service System. Such a metamodel is denoted as the PSS-IF Canonic Metamodel. For each domain-specific language, a viewpoint is defined on the basis of the PSS-IF Canonic Metamodel and captures only those parts of the canonic metamodel, which can be represented in the corresponding DSL. Furthermore, viewpoints are defined in such a way that they can be used for both reading from and writing to a model.

This makes it possible to achieve a transformation between two exemplary DSLs \textit{A} and \textit{B} through the following process:

\begin{enumerate}
\item Obtain the viewpoint for DSL \textit{A}.
\item Create an empty model.
\item Add data to this model from an external source, whose abstract syntax is that of DSL \textit{A}.
\item Obtain a viewpoint for DSL \textit{B}.
\item Create a new external target, conforming to the abstract syntax of DSL \textit{B}.
\item Extract data from the model through the viewpoint for DSL \textit{B} and write it to the target.
\end{enumerate}

Note that after step 3 the model actually contains data conforming to the PSS-IF Canonic Metamodel. This is because the data is added through a viewpoint which internally transforms it to conform to the canonic metamodel. This is why the data can, in step 6, be read directly with another viewpoint from the same model and is implicitly transformed to the abstract syntax of DSL \textit{B}.

\subsection{Parts of the PSS-IF Proof of Concept}

To realize the PSS-IF transformation approach, a number of concepts need to be defined. While these concepts are briefly described here on an abstract level, they also closely resemble the implementation, presented in detail in the next chapter. The concepts required are provided in the following sections.

\subsubsection{PSS-IF Metamodel}

A PSS-IF Metamodel consists of a multitude element types and data types. Each element type can be either a node type or an edge type. Both node and edge types can have attributes, and each attribute is bound to a particular data type. Associations between node types are defined through connection mappings, which are always bound to a certain edge type, i.e. an edge type can have many connection mappings, allowing it to associate different pairs of node types. Furthermore, inheritance can be defined among the node and edge types. It holds that attributes are inherited, while connection mappings are not. Additionally, node types can be defined to have edge type semantics under certain circumstances, so that chaining of associations is possible. Note that element and data type names are unique in the scope of a PSS-IF Metamodel and attribute names are unique in the scope of an element type. Finally, a PSS-IF Metamodel always contains a top-level node type called 'Node' and a top-level edge type called 'Edge', both of which define a set of predefined attributes.

\subsubsection{PSS-IF Model}

A PSS-IF Model consists of nodes and edges, both of which have attributes. A model does not have any knowledge of its own structure with respect to any PSS-IF Metamodel, i.e. all structural information in the model is on the level of incoming and outgoing edges and (untyped) attribute values. Since all structural information is held in the metamodel, accessing the same model with different metamodels will yield different results, as it is intended for the purpose of the transformations.

\subsubsection{Viewpoints and Transformations}

A \textit{transformation} is a function which, when applied to a metamodel, results in a transformed, slightly changed metamodel. This changed metamodel is denoted as a \textit{viewpoint} and represents a step in the direction of defining the way in which the PSS-IF Metamodel is seen from the perspective of a certain domain-specific language. The metamodel on which a transformation is applied is called \textit{base metamodel} of the resulting viewpoint. The viewpoint contains element types, i.e. node or edge Types, which, on demand, whenever they are used to operate on a model, apply the actual transformation of data. A new viewpoint can then be constructed by applying another transformation to an already existing viewpoint, which serves as it's base metamodel. Thus, through the chaining of transformations, different viewpoints can be constructed. This makes it possible to describe viewpoints for the different domain-specific languages the PSS-IF needs to support. It should be noted that, in general, the ordering of the application of transformations is of relevance. 

So far, the following transformations have been identified as sufficient for the definition of the viewpoints of all DSLs relevant for this interdisciplinary project:

\paragraph{Rename} The rename transformation is responsible for replacing the name of an existing element type with a new one in the scope of a viewpoint.

\paragraph{Alias} The alias transformation allows the definition of multiple element types within a viewpoint, all of which are mapped to a single element type within the base metamodel of the viewpoint.

\paragraph{Artificialize} The artificialize transformation is used for the creation of additional artificial entities in a model, when instantiating a node or an edge, respectively.

\paragraph{Hide} Applying this transformation to a metamodel removes a specific element type together with its subtypes, or a certain connection mapping of an edge type, from the resulting viewpoint.

\paragraph{Deinstantify} This transformation results in a viewpoint hiding all instances of a certain element type within a model. It is required to, in contrast to the hide transformation (see above), allow subtypes of the deinstantified type to be instantiated, while still hiding the instances of the deinstantified type itself.

\paragraph{Join} This transformation is used to internally create an edge between nodes, which are neighbours of the nods externally designated as the two ends of the respective edge. The actual neighbours used for the connection are determined in accordance with specific paths defined in the transformation.

\subsubsection{Generic Graph}

A generic graph is required as an abstraction for the concrete syntax of all external representations. The proof of concept implementation relies on a generic graph in which nodes and edges have a string designating their type, yet no further structural information.

In essence, the generic graph is used as follows: The data from an external representation is first read into a graph, which is then processed generically with the viewpoint of the respective DSL. In the same manner, when exporting a view, the model is first converted to a generic graph in accordance with the viewpoint of the respective DSL, and the generic graph is then serialized in accordance with the specifics of the concrete syntax of the DSL.

This approach has the following advantages:

\begin{itemize}
\item Separation of concerns: By using the graph it is possible to separate the handling of concrete and abstract syntax of each language from each other. The concrete syntax is handled by a specific utility, which rather relates to the syntax than to the language. For example, more than one language might be serialized to XML using the same utility. The abstract syntax of the DSL is meanwhile handled independently in accordance with the DSL's viewpoint.
\item Extension point for pre- and post-processing of data: It might be necessary, for certain DSLs, to perform certain pre- or post-processing modifications to the data, i.e. to normalize it. If such normalizations are out of the expressiveness of the operators of PSS-IF (i.e. are generic graph transformations), they can be applied directly onto the generic graph.
\end{itemize}

\subsubsection{Mappers}

Finally, mappers are utilities which encapsulate the whole transformation process to and from the PSS-IF Canonic Metamodel and a corresponding model. A mapper thus offers two functionalities, one for reading a model from an external representation and one for writing a model to an external representation. Each DSL has its own mapper and each mapper combines, in the appropriate order, the DSL's viewpoint creation, data transformation, any pre- and post-processing strategies, and the correct serialization utility.