\chapter{Implementation}
\label{chap:impl}

In the last chapter the conceptual approach to the realisation of the PSS-IF Proof-of-Concept was described. In particular, the latter sections of \chapref{chap:approach} describe the principles behind the chosen transformation method. In this context, this chapter focuses on the implementation of the framework. The chapter is structured as follows: \secref{sec:impl:technology} describes the technology stack used for the implementation. In the consequent \secref{sec:impl:principles} the adopted software development guiding principles are presented. Thereafter, \secref{sec:impl:structure} provides an overview of the project structure. \secref{sec:impl:components} follows with a description of each component of the PSS-IF PoC and, finally, \secref{sec:impl:process} describes the import and export processes and the collaboration of all components of the framework.

\section{Technology Stack}
\label{sec:impl:technology}

This section defines the technologies used for the implementation of the PSS-IF Proof-of-Concept. Each of the following subsections addresses a particular aspect of the technological stack. Note that all the software and tools used for the realization of the PoC are widely accepted industry standards.

\subsection{Programming Language}

To enable an easier and more rapid development of the prototype, a high-level programming language can better be utilized. Due to previous experience and know-how, the authors have chosen the Java Programming Language. Furthermore, the code is compliant with Java version 7, as distributed by Oracle. 

\subsection{Revision Control}

For improved parallelisation, better code maintenance and easier documentation, a distributed revision control can be used. While there are numerous alternatives, the authors have chosen GIT as a modern and powerful solution.

\subsection{Build Process and Dependency Management}

For the automated build process, as well as for the management of dependencies to external libraries, Apache Maven has been used.

\subsection{Test Framework}

For the execution of automated tests, JUnit, an industry standard framework, has been used.

\subsection{Used Libraries}

Next to the libraries provided by Oracle's Standard Java Runtime Environment, the following additional libraries have been used:

\paragraph{apache-compress} An API for the manipulation of different kinds of compressed files. In the scope of the Proof-of-Concept, the API is used in the Visio VSDX processing component, for the zipping and unzipping of VSDX files.

\paragraph{guava} Google Guava is a set of common libraries, mainly developed by Google. The package includes useful APIs for the manipulation of collections, the usage of function and predicates, and others.

\paragraph{junit} JUnit is an industry standard unit-testing framework.

\section{Guiding Principles}
\label{sec:impl:principles}

During the design and implementation of the PSS-IF PoC the authors have followed the principles and best-practices for software development. Some of the guiding principles were the following:

\paragraph{Standardization:} Usage of widely used and accepted industry-standard tools, technologies and formats, to render the produced solution more accessible to new developers and compliant to other pieces of software.

\paragraph{Patterns:} To maximize code quality and understandability, common architecture and design patterns have been utilized.

\paragraph{Object-Orientation:} The code is developed in accordance with the paradigms of the Java programming language -- it is mostly object-oriented and imperative.

\paragraph{Separation of Concerns:} The implementation follows the separation of concerns paradigm.

\section{Project Structure}
\label{sec:impl:structure}

Maven project with ... packaging. Stuff. TODO Note: maven modules do not neccessarily correspond to function modules: maven modules are coarse-grained. Each maven module contains one or more functional modules.

\section{Components}
\label{sec:impl:components}
 
TODO blah

\subsection{Core}

TODO

\subsection{Transformations}

TODO

\subsection{Generic Graph}

TODO

\subsection{I/O Mappers}

TODO

\subsection{Model Mappers}

TODO

\subsection{Microsoft Visio VSDX I/O}

TODO

\section{Import and Export Process}
\label{sec:impl:process}

TODO describe process, mit TEXT und BILD! 

\begin{itemize}
\item technology: java. maven... expplain why
\item guiding principles of development
\item project structure
\item core
\item transform
\item vsdx
\item viz?
\end{itemize}